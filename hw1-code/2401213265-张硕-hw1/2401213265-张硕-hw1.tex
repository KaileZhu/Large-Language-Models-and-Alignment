\documentclass{article}
\usepackage[UTF8]{ctex}
\usepackage{enumitem}
\usepackage{xcolor}
\usepackage{amsmath,amssymb}
\usepackage{algorithm,algpseudocode}
\usepackage{hyperref}
\usepackage{booktabs}
\usepackage{tabularx}

\hypersetup{colorlinks=true,urlcolor=blue}
\renewcommand{\algorithmicrequire}{\textbf{输入:}}
\renewcommand{\algorithmicensure}{\textbf{输出:}}

\title{BPE算法详解与LLM Tokenizer训练流程}
\author{Your Name}
\date{\today}

\begin{document}
\maketitle

\section{BPE算法(Byte Pair Encoding)}
\subsection{算法背景}
BPE是一种\textbf{数据压缩算法},由Philip Gage于1994年提出。在NLP领域,Sennrich等人(2015)首次将其应用于子词分词,主要解决:
\begin{itemize}
    \item 开放词汇表问题(Open Vocabulary)
    \item 罕见词表示问题(Rare Word Representation)
    \item 多语言统一处理需求
\end{itemize}

\subsection{数学形式化定义}
给定语料库$D$,初始化字符集$V_0$,迭代执行以下操作:
\[
(v_i, v_j) = \mathop{\arg\max}\limits_{(x,y) \in P} \text{count}(x,y)
\]
其中:
\begin{itemize}
    \item $P$为当前所有可能相邻符号对集合
    \item $\text{count}(x,y)$表示$(x,y)$在$D$中的共现频率
\end{itemize}
合并操作更新词汇表:
\[
V_{k+1} = V_k \cup \{v_i v_j\} \setminus \{v_i, v_j\}
\]

\section{BPE核心算法流程}
\begin{algorithm}[H]
\caption{BPE训练过程}
\begin{algorithmic}[1]
\Require 原始文本语料$D$,目标词汇量$K$
\Ensure BPE词汇表$V$
\State 预处理:将$D$中所有单词添加终止符\texttt{</w>}
\State 初始化$V_0 \gets$ 所有唯一字符 + \texttt{</w>}
\State 统计所有单词的词频$f(w)$
\For{$k=1$ to $K-|V_0|$}
    \State 统计所有相邻符号对频率$f(pair)$
    \State 选择最高频pair $(x,y)$
    \State 将合并操作$x + y \rightarrow z$加入合并规则表
    \State 更新$V_k \gets V_{k-1} \cup \{z\}$
    \State 更新语料中所有$(x,y)$出现位置为$z$
\EndFor
\end{algorithmic}
\end{algorithm}

\subsection{详细步骤解释}
\begin{enumerate}[label=\textbf{步骤 \arabic*},leftmargin=*]
    \item \textbf{符号化预处理}
    \begin{itemize}
        \item 将文本转换为Unicode编码(处理多语言)
        \item 添加单词边界标记:\texttt{low} $\Rightarrow$ \texttt{l o w </w>}
        \item 示例词汇表:$V_0 = \{\texttt{l}, \texttt{o}, \texttt{w}, \texttt{</w>}\}$
    \end{itemize}
    
    \item \textbf{频率统计阶段}
    \begin{itemize}
        \item 构建共现矩阵:$\text{count}(x,y) = \sum_{w \in D} f(w) \cdot N_w(x,y)$
        \item 其中$N_w(x,y)$表示单词$w$中$(x,y)$的出现次数
    \end{itemize}
    
    \item \textbf{动态合并过程}
    \begin{table}[H]
    \centering
    \caption{BPE合并过程示例}
    \begin{tabular}{cll}
    \toprule
    迭代次数 & 合并操作 & 新词汇表 \\
    \midrule
    1 & (o, w) $\rightarrow$ ow & \texttt{ow, l, o, w, </w>} \\
    2 & (l, ow) $\rightarrow$ low & \texttt{low, ow, l, o, w, </w>} \\
    3 & (e, r) $\rightarrow$ er & \texttt{er, low, ow, ...} \\
    \bottomrule
    \end{tabular}
    \end{table}
    
    \item \textbf{终止条件}
    \begin{itemize}
        \item 预设条件:达到目标词汇量$K$(典型值:32k, 50k)
        \item 动态条件:$\max f(pair) < \theta$(阈值$\theta$通常设为2)
    \end{itemize}
\end{enumerate}

\section{LLM Tokenizer训练流程}
\subsection{完整训练架构}
\begin{figure}[H]
\centering
\includegraphics[width=0.8\textwidth]{bpe_pipeline.png}
\caption{BPE Tokenizer训练流程图}
\label{fig:pipe}
\end{figure}

\subsection{关键技术细节}
\begin{itemize}
    \item \textbf{字节级处理}(GPT-2方案)
    \begin{itemize}
        \item 将文本转换为UTF-8字节序列(256个基础token)
        \item 优点:完全消除未知字符问题
        \item 示例:汉字「中」$\rightarrow$ \texttt{E4 B8 AD}
    \end{itemize}
    
    \item \textbf{子词正则化}
    \begin{itemize}
        \item 使用Unigram语言模型进行采样
        \item 实现概率化的分词结果
    \end{itemize}
    
    \item \textbf{合并策略优化}
    \begin{equation}
    \text{Score}(x,y) = \frac{\text{count}(x,y)}{\text{count}(x) \times \text{count}(y)}
    \end{equation}
    选择互信息最大的符号对进行合并
\end{itemize}

\section{实现对比分析}
\begin{table}[H]
\centering
\caption{BPE变体对比}
\begin{tabularx}{\textwidth}{lXXX}
\toprule
\textbf{类型} & \textbf{BPE} & \textbf{WordPiece} & \textbf{Unigram} \\
\midrule
合并策略 & 频率优先 & 最大似然估计 & 概率删除 \\
训练方式 & 贪婪合并 & 迭代EM算法 & 动态规划 \\
处理未知词 & 子词分解 & 同左 & 同左 \\
典型应用 & GPT系列 & BERT & XLNet \\
\bottomrule
\end{tabularx}
\end{table}

\section*{附录:代码实现}
\begin{itemize}
    \item Hugging Face实现核心逻辑:
    \begin{verbatim}
class BPE:
    def train(self, texts, vocab_size):
        merges = []
        vocab = self._build_initial_vocab(texts)
        while len(vocab) < vocab_size:
            pairs = self._get_pairs_with_freq(texts)
            if not pairs:
                break
            best_pair = max(pairs, key=pairs.get)
            texts = self._merge_pair(best_pair, texts)
            merges.append(best_pair)
            vocab.add(''.join(best_pair))
        return merges
    \end{verbatim}
\end{itemize}
\end{document}